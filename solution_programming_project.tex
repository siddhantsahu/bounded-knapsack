\documentclass{article}
\usepackage[utf8]{inputenc}
\usepackage{amsmath,amssymb,amsthm}
\usepackage[margin=1in]{geometry}

\title{CS 6363.004 Algorithms: Programming Project}
\author{Siddhant Sahu}

\setlength{\parindent}{0em}

\begin{document}
\maketitle

\paragraph{Problem Formulation}
We define $m(i, w)$ to be the maximum profit that can be generated using the first $i$ items and $w$ units of gold, where $i$ and $w$ are both non-negative integers. Clearly, $m(0, w) = 0$ for all $w$. We can't make any profit with 0 items. The recurrence is:

\[ m(i, w) =
\begin{cases}
\max (m(i-1, w-k \cdot w_i) + k \cdot p_i - min(c_i, f_i \cdot (n_i - k)))       & \quad \text{if } 0 \leq k < n_i\\
\max (m(i-1, w-k \cdot w_i) + k \cdot p_i)  & \quad \text{if } n_i \leq k \leq x_i
\end{cases}
\]

where $k$ represents the quantity (i.e. number of copies) of item $i$.
$$\ 0 \leq k \leq min\left(x_i, \left\lfloor\frac{w}{w_i}\right\rfloor\right)$$

\paragraph{Feasibility}
Proof by induction on $i$. We claim that there exists a solution for $m(i, w)$.\\
Basis Step: Let $i = 0$. We can not make any profit with 0 items. Thus, $m(0, w) = 0$, which is correct. This establishes the basis step.\\
Inductive Step: For any $(i, w)$ and $\ 0 \leq k \leq min\left(x_i, \left\lfloor\frac{w}{w_i}\right\rfloor\right)$, we have two cases as mentioned in the recursion. In both cases, we compute $m(i-1, w-k \cdot w_i)$. Since, $i-1 < i$, by induction hypothesis, $m(i-1, w-k \cdot w_i)$ is feasible. This completes the inductive step and therefore, the feasibility proof.

\paragraph{Optimality}
Let $Opt(i, w)$ be an optimal solution for the problem by using the first $i$ items and $w$ units of gold. We claim that $m(i, w) \leq Opt(i, w)$. Proof by induction on $i$.\\
Basis Step: Let $i = 0$. We can not make any profit with 0 items. Thus, $m(0, w) = 0 = Opt(0, w)$, which is correct.\\
Inductive Step: Consider $i > 0$ and $0 \leq k \leq min\left(x_i, \left\lfloor\frac{w}{w_i}\right\rfloor\right)$. We have two cases here.
\begin{itemize}
	\item Case 1: $0 \leq k \leq n_i$, i.e. the jeweler has to pay a fine. Suppose $Opt(i, w)$ selects $k'$ quantities of item $i$. Thus, $Opt(i, w) = Opt(i-1, w-k' \cdot w_i) + k' \cdot p_i - min(c_i, f_i \cdot(n_i - k'))$
	\begin{align*}
	m(i, w) &= \max\limits_{0 \leq k \leq min\left(x_i, \left\lfloor\frac{w}{w_i}\right\rfloor\right)}(m(i-1, w-k \cdot w_i) + k \cdot p_i - min(c_i, f_i \cdot(n_i - k)))\\
	& \leq  m(i-1, w-k' \cdot w_i) + k' \cdot p_i - min(c_i, f_i \cdot(n_i - k'))\\
	& \leq Opt(i-1, w-k' \cdot w_i) + k' \cdot p_i - min(c_i, f_i \cdot(n_i - k'))\\
	& \leq Opt(i, w)
	\end{align*}
	Since $i-1 < i$, $m(i-1,w-k' \cdot w_i) \leq Opt(i-1,w-k' \cdot w_i)$ by induction hypothesis.
	
	\item Case 2: $n_i \leq k \leq x_i$. Proof is almost the same as case 1, without the fines.
\end{itemize}

This completes the proof for optimality.

\paragraph{Running Time Analysis}
The worst-case running time is $O(NkW)$, where $k$ is the maximum value in $x[1..n]$. There are three loops running -- the outermost one loops over all the items ($N$), the inner loop goes over all possible weights (units of gold, $W$) and the innermost loop goes over all possible quantities of the current item that can be selected.

\end{document}

\documentclass{article}
\usepackage[utf8]{inputenc}
\usepackage{amsmath,amssymb,amsthm}
\usepackage[margin=0.5in]{geometry}
\usepackage{clrscode3e}

\title{CS 6363.004 Algorithms: Programming Project}
\author{Siddhant Sahu}
\date{April 1, 2018}

\setlength{\parindent}{0em}
\newcommand{\floor}[1]{\left\lfloor #1 \right\rfloor}
\newcommand{\itemattr}[1]{\attrib{\id{items[i]}}{#1}}
\DeclareMathOperator*{\argmax}{argmax}

\begin{document}
\maketitle

\section{Finding Maximum Profit}

\subsection{Recurrence}
We define $m(i, w)$ to be the maximum profit that can be generated using the first $i$ items and $w$ units of gold, where $i$ and $w$ are both non-negative integers. Clearly, the jeweler can't make any profit with 0 items. So, $m(0, w) = 0$. 
When the jeweler is unable to meet the minimum quantity ($k < n_i$) according to the contract, a fine $\min (c_i, f_i \cdot (n_i - k)))$ is levied. The recurrence for finding maximum profit is:

\[ m(i, w) =
\begin{cases}
\max (m(i-1, w-k \cdot w_i) + k \cdot p_i - \min (c_i, f_i \cdot (n_i - k)))       & \quad \text{if } 0 \leq k < n_i\\
\max (m(i-1, w-k \cdot w_i) + k \cdot p_i)  & \quad \text{if } n_i \leq k \leq x_i
\end{cases}
\]

where $k$ represents the quantity (i.e. number of copies) of item $i$.
$$\ 0 \leq k \leq min\left(x_i, \left\lfloor\frac{w}{w_i}\right\rfloor\right)$$

\subsection{Proof of correctness}

\paragraph{Feasibility}
Proof by induction on $i$. We claim that there exists a solution for $m(i, w)$.\\
Basis Step: Let $i = 0$. We can not make any profit with 0 items. Thus, $m(0, w) = 0$, which is correct. This establishes the basis step.\\
Inductive Step: For any $(i, w)$ and $\ 0 \leq k \leq min\left(x_i, \left\lfloor\frac{w}{w_i}\right\rfloor\right)$, we have two cases as mentioned in the recursion. In both cases, we compute $m(i-1, w-k \cdot w_i)$. Since, $i-1 < i$, by induction hypothesis, $m(i-1, w-k \cdot w_i)$ is feasible. This completes the inductive step and therefore, the feasibility proof.

\paragraph{Optimality}
Let $Opt(i, w)$ be an optimal solution for the problem by using the first $i$ items and $w$ units of gold. We claim that $m(i, w) \geq Opt(i, w)$. Proof by induction on $i$.\\
Basis Step: Let $i = 0$. We can not make any profit with 0 items. Thus, $m(0, w) = 0 = Opt(0, w)$, which is correct.\\
Inductive Step: Consider $i > 0$ and $0 \leq k \leq min\left(x_i, \left\lfloor\frac{w}{w_i}\right\rfloor\right)$. We have two cases here.
\begin{itemize}
	\item Case 1: $0 \leq k < n_i$, i.e. the jeweler has to pay a fine. Suppose $Opt(i, w)$ selects $k'$ quantities of item $i$. Thus, $Opt(i, w) = Opt(i-1, w-k' \cdot w_i) + k' \cdot p_i - \min(c_i, f_i \cdot(n_i - k'))$
	\begin{align*}
	m(i, w) &= \max\limits_{0 \leq k \leq min\left(x_i, \left\lfloor\frac{w}{w_i}\right\rfloor\right)}(m(i-1, w-k \cdot w_i) + k \cdot p_i - \min(c_i, f_i \cdot(n_i - k)))\\
	& \geq  m(i-1, w-k' \cdot w_i) + k' \cdot p_i - \min(c_i, f_i \cdot(n_i - k'))\\
	& \geq Opt(i-1, w-k' \cdot w_i) + k' \cdot p_i - \min(c_i, f_i \cdot(n_i - k'))\\
	& \geq Opt(i, w)
	\end{align*}
	Since $i-1 < i$, $m(i-1,w-k' \cdot w_i) \geq Opt(i-1,w-k' \cdot w_i)$ by induction hypothesis.
	
	\item Case 2: $n_i \leq k \leq x_i$. Proof is almost the same as case 1, without the fines.
\end{itemize}

This completes the proof for optimality.

\subsection{Pseudocode}

The input to $\proc{Max-Profit-Calculator}$ is the total units of gold available $W$ and a list of items $\id{items}$ with attributes $w$, $p$, $n$, $x$, $f$ and $c$. Attribute naming convention is consistent with the problem description.

\begin{codebox}
\Procname{$\proc{Max-Profit-Calculator}(W, \id{items})$}
\li $N \gets \attrib{\id{items}}{length}$
\li let $m[0 \twodots N, 0 \twodots W]$ be a new table
\li \For $w = 0$ \To $W$ \Do
\li 	$m[0, w] \gets 0$
	\End
\li \For $i = 1$ \To $N$ \Do
\li 	\For $w = 1$ \To $W$ \Do
\li 		$m[i, w] \gets -\infty$
\li 		$\id{quantity} \gets \min(\itemattr{x}, \floor{w/\itemattr{w}})$
\li			\For $k = 0$ \To $\id{quantity}$ \Do
\li 			$q \gets m[i-1, w-k \cdot \itemattr{w}] + k \cdot \itemattr{p}$
\li 			\If $k < \itemattr{n}$ \Then
\li 				$q \gets q - \min(\itemattr{c}, \itemattr{f} \cdot (\itemattr{n} - k))$
				\End
\li 			\If $q > m[i, w]$ \Then
\li 				$m[i, w] \gets q$
				\End
			\End		
		\End
	\End
\li \Return $m$
\end{codebox}

Maximum profit is $m[N, W]$. A quick inspection tells us the running time is $O(kNW)$, where $k$ is the maximum value in $x[1 \twodots n]$. If $k=N$, running time is $O(N^2W)$.

\section{Counting Number of Solutions}

\subsection{Recurrence}
We define $c(i, w)$ to be the number of solutions that generate the maximum profit $m(i, w)$. With 0 items, the solution set is empty. With one item, there is exactly one solution. Thus, the recurrence is:

\[ c(i, w) =
\begin{cases}
\sum_{k \in \argmax{m(i, w)}} c(i-1, w-k \cdot w_i) & \quad \text{if } i \geq 2\\
1 & \quad \text{if } i = 1\\
0 & \quad \text{if } i = 0
\end{cases}
\]

\subsection{Pseudocode}
The input to $\proc{Count-Solutions}$ is $m$, the table returned from $\proc{Max-Profit-Calculator}$, $W$ and a list of $\id{items}$.

\begin{codebox}
\Procname{$\proc{Count-Solutions}(m, W, \id{items})$}
\li $N \gets \attrib{\id{items}}{length}$
\li let $c[0 \twodots N, 0 \twodots W]$ be a new table
\li \For $w = 0$ \To $W$ \Do
\li 	$c[0, w] \gets 0$
	\End
\li \For $w = 0$ \To $W$ \Do
\li 	$c[1, w] \gets 1$
	\End
\li \For $i = 2$ \To $N$ \Do
\li 	\For $w = 1$ \To $W$ \Do
\li 		$\id{quantity} \gets \min(\itemattr{x}, \floor{w/\itemattr{w}})$
\li			\For $k = 0$ \To $\id{quantity}$ \Do
\li 			$q \gets m[i-1, w-k \cdot \itemattr{w}] + k \cdot \itemattr{p}$
\li 			\If $k < \itemattr{n}$ \Then
\li 				$q \gets q - \min(\itemattr{c}, \itemattr{f} \cdot (\itemattr{n} - k))$
				\End
\li 			\If $q \isequal m[i, w]$ \Then
\li 				$c[i, w] \gets c[i, w] + c[i-1, w-k \cdot \itemattr{w}]$
			\End
		\End		
	\End
\End
\li \Return $c$
\end{codebox}

Total number of solutions to the problem is $c[N, W]$. The running time is same as $\proc{Max-Profit-Calculator}$, i.e. $O(kNW)$ or $O(N^2W)$ if $k = N$.

\end{document}